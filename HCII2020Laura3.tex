% This is samplepaper.tex, a sample chapter demonstrating the
% LLNCS macro package for Springer Computer Science proceedings;
% Version 2.20 of 2017/10/04
%
\documentclass[runningheads]{llncs}
%
\usepackage{fixltx2e}
\usepackage[american]{babel}
\usepackage[utf8]{inputenc}
\usepackage{csquotes}
\usepackage{graphicx}
\usepackage{xcolor}
\usepackage{hyperref}
\usepackage{url}
\usepackage[T1]{fontenc}
\usepackage{lmodern}
\usepackage{microtype}
\usepackage{eurosym}
\usepackage{biblatex}
\addbibresource{bibliography.bib}
\addbibresource{rpackages.bib}
% Used for displaying a sample figure. If possible, figure files should
% be included in EPS format.
%
% If you use the hyperref package, please uncomment the following line
% to display URLs in blue roman font according to Springer's eBook style:
%\renewcommand\UrlFont{\color{blue}\rmfamily}
\hypersetup{breaklinks=true,
            bookmarks=true,
            pdfauthor={},
            pdftitle={Netlogo vs.~Julia programming: The best way to simulate opinion formation},
            colorlinks=true,
            citecolor=blue,
            urlcolor=blue,
            linkcolor=magenta,
            pdfborder={0 0 0}}
\urlstyle{same}


\newcommand{\eg}{e.\,g.,\ }
\newcommand{\ie}{i.\,e.,\ }
%\setkeys{Gin}{width=\maxwidth,height=\maxheight,keepaspectratio}

\IfFileExists{upquote.sty}{\usepackage{upquote}}{}
% use microtype if available
\IfFileExists{microtype.sty}{%
\usepackage{microtype}
\UseMicrotypeSet[protrusion]{basicmath} % disable protrusion for tt fonts
}{}

\providecommand{\tightlist}{%
  \setlength{\itemsep}{0pt}\setlength{\parskip}{0pt}}

\begin{document}
%
\title{Netlogo vs.~Julia programming: The best way to simulate opinion
formation}
%
%\titlerunning{Abbreviated paper title}
% If the paper title is too long for the running head, you can set
% an abbreviated paper title here
%
\author{First Author\inst{1}\orcidID{0000-1111-2222-3333} \and
Second Author\inst{2,3}\orcidID{1111-2222-3333-4444} \and
Third Author\inst{3}\orcidID{2222--3333-4444-5555}}
%
\authorrunning{F. Author et al.}
% First names are abbreviated in the running head.
% If there are more than two authors, 'et al.' is used.
%
\institute{Springer Heidelberg, Tiergartenstr. 17, 69121 Heidelberg, Germany
\email{lncs@springer.com}\\
\url{http://www.springer.com/gp/computer-science/lncs}\\
\email{\{abc,lncs\}@uni-heidelberg.de}}
%


%
\maketitle              % typeset the header of the contribution
%
\begin{abstract}

	\keywords{Simulation \and Agent-based modells \and Programming languages
\and Netlogo \and Julia programming.}
\end{abstract}
%
%
%
We used the following packages to create this document:
\texttt{knitr}~\autocite{R-knitr},
\texttt{tidyverse}~\autocite{R-tidyverse},
\texttt{rmdformats}~\autocite{R-rmdformats},
\texttt{kableExtra}~\autocite{R-kableExtra},
\texttt{scales}~\autocite{R-scales}, \texttt{psych}~\autocite{R-psych},
\texttt{rmdtemplates}~\autocite{R-rmdtemplates}.

\hypertarget{introduction}{%
\section{Introduction}\label{introduction}}

When humans interact with each other or with digitized technology we
speak of complex systems. The interaction of humans in such systems, for
example in opinion-forming processes, leads to consequences that we
cannot yet overlook or understand. An important component of
socio-technical complex systems are single individuals that appear as
human-in-the-loop~\autocite{Valdez2018human}. To look at the humans,
their interactions and the resulting overall behavior, different
simulation approaches using different programming languages are chosen.

\hypertarget{related-work}{%
\section{Related Work}\label{related-work}}

When examining opinion-forming processes, we look at a complex system.
Such complex systems lead to emergent phenomena. Complex systems and
emergent phenomena are difficult to understand, because while it is easy
to observe the individual system components, the resulting overall
system cannot be considered as the sum of its parts. Instead,
understanding the system behavior requires more than understanding the
individual parts of the system~\autocite{Valdez2018human}.

To analyse complex systems we need a suitable approach, such as
simulations, which enable to model the individual parts of a system and
thus make the overall behavior visible. For the simulation of complex
systems, Agent-based models are very well suited~\autocite{Epstein2007}.
Agent-based models always consist of the agents or individuals and the
environment in which the agents
reside~\autocite{Bonabeau2002Agentbased}.

To create agent-based models, Netlogo~\autocite{Wilensky1999} is the
language most commonly used. Nevertheless, there are some other
programming languages that are also suitable for creating agent-based
models and that seem to be partly more intuitive, at least for people
with programming experience. Therefore, in this study we compare two
programming languages with respect to their suitability for creating
agent-based models.

\hypertarget{method}{%
\section{Method}\label{method}}

Using two different programming languages (Julia language and Netlogo),
we created two identical agent-based models that simulate opinion
formation. Since our primary aim was to find out whether agent-based
models could be implemented equally well in the two programming
languages, we chose the most basic model of opinion-forming: bounded
rationality.

In our bounded rationality model, we defined the maximum number of
agents, the maximum steps of the simulation, the seed and an epsilon in
the beginning. The epsilon indicates how different the opinions of two
people can be, so that they still include the other person's opinion in
their opinion making. We further defined from the beginning, that each
agent has an (floating) opinion between 0 and 1. In each simulation
step, every agent compares his opinion with the opinion of an other
agent. For example, if Anna compares her opinion with Ralf and the
distance between the opinion of Anna and Ralf is smaller than the
defined epsilon, then the two converge in their opinions.

We build the agent based models using either the Julia programming
language or Netlogo. For the following analysis of the results we used R
Markdown.

\hypertarget{what-do-we-compare}{%
\subsection{What do we compare}\label{what-do-we-compare}}

To find out whether both programming languages are euqually suitable to
simulate our bounded rationality model, we look at severeal measurable
criteria. These criteria include the outcomes and performance of both
models. They further include how many lines of code are necessary to
program the simulation. Another aspect, that we take into consideration,
is, if learning Julia and Netlogo is equally difficult. For this aspect
we consider both computer scientists who are familiar with other
programming languages and a person who has no previous experience with
programming languages. We further compare the explorability and
scalability of both languages.

\hypertarget{results}{%
\section{Results}\label{results}}

Following, we first present the outcome of the bounded rationality model
created with julia. Afterwards we show, if the model created with
Netlogo showed the same or different results. Based on these results, we
compare the two considered programming languages and show their
advantages and disadvantages.

\hypertarget{julia-simulation}{%
\subsection{Julia simulation}\label{julia-simulation}}

\hypertarget{comparison-of-julia-and-netlogo}{%
\subsection{Comparison of Julia and
Netlogo}\label{comparison-of-julia-and-netlogo}}

When comparing both programming languages to create an agent-based model
that simulates the bounded rationality model, Julia proved to be a
faster language less code lines were required to simulare our model. In
addition, Julia turned out to be easier to learn for people with
previous programming skills. In contrast, Netlogo proved to be a
language that is easier to learn for people without programming skills.

\hypertarget{discussion}{%
\section{Discussion}\label{discussion}}

In our study, no language turned out to be the perfect programming
language for creating agent-based models, but the choice of language
seems to be a trade-off between various advantages and disadvantages. Of
course, we have only focused on one very simple bounded rationality
model, so that we would have to create further simulations with both
languages to be able to make statements about the generality.

\hypertarget{conclusion-and-outlook}{%
\section{Conclusion and outlook}\label{conclusion-and-outlook}}

The results of our research have shown that Julia is the better choice
for creating an agent-based model in some aspects, but also that Netlogo
is better suited in some aspects. In the future, we plan to create more
agent-based models using both programming languages, thus extending our
comparison.


%
% ---- Bibliography ----
%
% BibTeX users should specify bibliography style 'splncs04'.
% References will then be sorted and formatted in the correct style.
%
%\bibliographystyle{splncs04}
%\bibliography{bibliography,rpackages}
\printbibliography



\end{document}
